\documentclass{classrep}
\usepackage[utf8]{inputenc}
\usepackage{color}

\studycycle{Informatyka, studia niestacjonarne, I st.}
\coursesemester{VI}

\coursename{Komputerowe systemy rozpoznawania}
\courseyear{2019}

\courseteacher{dr hab. inż. Adam Niewiadomski}
\coursegroup{Niedziela, 12:00}

\author{
  \studentinfo{Konrad Jachimstal}{211807} \and
  \studentinfo{ Patryk Janicki}{211951}
}

\title{Zadanie 1: Ekstrakcja cech, miary podobieństwa, klasyfikacja}

\begin{document}
\maketitle

\section{Cel}
{Celem zadania jest zbadanie wpływu ekstrakcji cech oraz wykorzystanych miar podobieństwa w procesie klasyfikacji tekstu. 
Klasyfikacja tekstów ma zostać zrealizowania z wykorzystaniem algorytmu najbliższych sąsiadów (KNN).}

\section{Wprowadzenie}
Klasyfikacja tekstów, która odbywa się za pomocą algorytmu KNN. Polega ona na przypisaniu tekstu do odpowiedniej
kategorii. Odbywa się to na podstawie watrości poszczególnych wyekstrachowanych cech opisanych poniżej.
\subsection{Wykorzystane ekstraktory cech}

\subsubsection{Liczba słów kluczowych, które wystąpiły}
Występowanie określonych słów kluczowych w tekście,
nie uwzględnia liczności słów. Cecha to liczba słów kluczowych,
określająca ile słów kluczowych z całej puli wystąpiło w tekście.

\subsubsection{Suma wystąpień słów kluczowych //TODO zmienić opis}
Suma wystąpień każdego słowa kluczowego w tekście. Cecha to suma wystąpień wszystkich słów kluczowych w tekście.

\begin{equation}
    F=S_{1} + ... + S_{n}
\end{equation}
gdzie $S$ - liczba wystąpień słowa kluczowego.


\subsubsection{Gęstość występowania słów kluczowych}
Gęstość wyrażona jako iloraz sumy wystąpień słów kluczowych i liczby wszystkich słów w tekście.
Cecha przyjmuje wartości liczbowe z zakresu od 0 do 1.
    \begin{equation}
      F=\frac{S}{L}
    \end{equation}
gdzie $S$ - suma wystąpień słów kluczowych, $L$ - liczba wszystkich słów w tekście.


\subsubsection{Średnia odległość słów kluczowych od początku tekstu}
Średnia odległość słów kluczowych wyrażona jako iloraz sumy odległości słów kluczowych od początku tekstu i liczby
słów kluczowych w tekście. Cecha to suma wartości liczbowych odzwierciedlających odległość danego słowa kluczowego
od początku tekstu, gdzie pojedyncze słowo odpowiada odległości równej jeden.
\begin{equation}
    F=\frac{S_{1} + ... + S_{n}}{L}
\end{equation}
gdzie $S$ - odległość słowa kluczowego od początku tekstu, $L$ -  liczba słów kluczowych w tekście.



\subsubsection{Liczba wszystkich słów}
Określenie liczby wszystkich słów występujących w tekście. 
Cecha to wartość liczbowa równa liczbie słów w tekście.

\subsubsection{Rozproszenie słów kluczowych}
Rozproszenie słów kluczowych wyrażone jako iloraz sumy odległości pomiędzy słowami kluczowymi i iloczyn liczby
wszystkich słów w tekście oraz liczby słów kluczowych w tekście.

Możemy podjąć próbę i założyć, że większe nagromadzenie słów kluczowych na przestrzeni całego tekstu pozwala
stwierdzić, czy określone słowa kluczowe mają powiązanie z tematem tekstu, a nie stanowią wzmianki czy wstępu do
tekstu. Ekstrakcja ma pozwolić na określenie rozrzutu słów kluczowych w tekście. Cecha jest określona na podstawie
sumy odległości pomiędzy słowami kluczowymi z uwzględnieniem gęstości ich występowania.
\begin{equation}
    F=\frac{E_{d}}{N*N_{k}}
\end{equation}
gdzie $E_{d}$ - suma odległości pomiędzy słowami kluczowymi, $N$ - liczba wszystkich słów w tekście,
$N_{k}$ - liczba słów kluczowych w tekście.



\subsubsection{Występowanie określonych podciągów //TODO zaimplementować}
\dots

\subsubsection{Częstość słów (ang. Term Frequency) //TODO zaimplementować}
Częstość słów wtyrażona jako iloraz liczby wytąpień słowa kluczowegoh przez ilość słów w dokumencie. Cecha to
częstotliwość wytępowania słowa kluczowego.
\begin{equation}
    F=\frac{K}{W}
\end{equation}
gdzie $K$ - liczba wystąpień danego słowa kluczowego w dokumencie, $W$ - liczba wszystkich słów w dokumencie

\subsubsection{IDF (ang. inverse document frequency) //TODO zaimplementować}
IDF wyrażony jest jako iloraz liczby dokumentów poddawanych badaniu przez liczbę dokumentów w których dane słowo
kluczowe wystąpiło przynajmniej raz.

\begin{equation}
    F=log \frac{W}{D}
\end{equation}
gdzie $W$ - liczba wszystkich dokumentów, $D$ - liczba dokumentów w których wystąpiło słowo kluczowe

\subsubsection{TF-IDF //TODO zaimplementować}
TF-IDF jest to stosunek częstości słów do częstości występowanie danego słowa we wszystkich dokumentach.
Worażona jako iloczyn częstości słów (TF) oraz IDF.
\begin{equation}
    F={TF}\cdot{IDF}
\end{equation}
gdzie $TF$ - częstość słów, $IDF$ - częstość występowania na tle innych dokuemntów.

\newline \newline
{\color{blue}
We wprowadzeniu należy zaprezentować całą teorię potrzebną do realizacji
zadania (przy czym należy tu ograniczyć się wyłącznie do tego, co było
wykorzystane) tak aby osoba, która nigdy wcześniej nie zetknęła się z tą
tematyką, potrafiła zrozumieć dalszy opis. Część ta powinna wprowadzać
wszystkie wykorzystywane wzory, oznaczenia itp., do których należy się
odwoływać w dalszej części niniejszgo sprawozdania. Zamieszczony tu własny
opis teorii (a nie skopiowany!) należy poprzeć odwołaniami bibliograficznymi
do literatury zamieszczonej na końcu. }

\section{Opis implementacji}
{\color{blue}
Należy tu zamieścić krótki i zwięzły opis zaprojektowanych klas oraz powiązań
między nimi. Powinien się tu również znaleźć diagram UML  (diagram klas)
prezentujący najistotniejsze elementy stworzonej aplikacji. Należy także
podać, w jakim języku programowania została stworzona aplikacja. }

\section{Materiały i metody}
{\color{blue}
W tym miejscu należy opisać, jak przeprowadzone zostały wszystkie badania,
których wyniki i dyskusja zamieszczane są w dalszych sekcjach. Opis ten
powinien być na tyle dokładny, aby osoba czytająca go potrafiła wszystkie
przeprowadzone badania samodzielnie powtórzyć w celu zweryfikowania ich
poprawności (a zatem m.in. należy zamieścić tu opis architektury sieci,
wartości współczynników użytych w kolejnych eksperymentach, sposób
inicjalizacji wag, metodę uczenia itp. oraz informacje o danych, na których
prowadzone były badania). Przy opisie należy odwoływać się i stosować do
opisanych w sekcji drugiej wzorów i oznaczeń, a także w jasny sposób opisać
cel konkretnego testu. Najlepiej byłoby wyraźnie wyszczególnić (ponumerować)
poszczególne eksperymenty tak, aby łatwo było się do nich odwoływać dalej.}

\section{Wyniki}
{\color{blue}
W tej sekcji należy zaprezentować, dla każdego przeprowadzonego eksperymentu,
kompletny zestaw wyników w postaci tabel, wykresów itp. Powinny być one tak
ponazywane, aby było wiadomo, do czego się odnoszą. Wszystkie tabele i wykresy
należy oczywiście opisać (opisać co jest na osiach, w kolumnach itd.) stosując
się do przyjętych wcześniej oznaczeń. Nie należy tu komentować i interpretować
wyników, gdyż miejsce na to jest w kolejnej sekcji. Tu również dobrze jest
wprowadzić oznaczenia (tabel, wykresów) aby móc się do nich odwoływać
poniżej.}

\section{Dyskusja}
{\color{blue}
Sekcja ta powinna zawierać dokładną interpretację uzyskanych wyników
eksperymentów wraz ze szczegółowymi wnioskami z nich płynącymi. Najcenniejsze
są, rzecz jasna, wnioski o charakterze uniwersalnym, które mogą być istotne
przy innych, podobnych zadaniach. Należy również omówić i wyjaśnić wszystkie
napotakane problemy (jeśli takie były). Każdy wniosek powinien mieć poparcie
we wcześniej przeprowadzonych eksperymentach (odwołania do konkretnych
wyników). Jest to jedna z najważniejszych sekcji tego sprawozdania, gdyż
prezentuje poziom zrozumienia badanego problemu.}
\section{Wnioski}
{\color{blue}W tej, przedostatniej, sekcji należy zamieścić podsumowanie
najważniejszych wniosków z sekcji poprzedniej. Najlepiej jest je po prostu
wypunktować. Znów, tak jak poprzednio, najistotniejsze są wnioski o
charakterze uniwersalnym.}


\begin{thebibliography}{0}
\end{thebibliography}
{\color{blue} 
Na końcu należy obowiązkowo podać cytowaną w sprawozdaniu
literaturę, z której grupa korzystała w trakcie prac nad zadaniem (przykład na
końcu szablonu)}
\end{document}

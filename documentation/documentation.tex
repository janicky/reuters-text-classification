\documentclass{classrep}
\usepackage[utf8]{inputenc}
\usepackage{color}
\usepackage{amsmath}
\usepackage[T1]{fontenc}

\studycycle{Informatyka, studia niestacjonarne, I st.}
\coursesemester{VI}

\coursename{Komputerowe systemy rozpoznawania}
\courseyear{2019}

\courseteacher{dr hab. inż. Adam Niewiadomski}
\coursegroup{Niedziela, 12:00}

\author{
  \studentinfo{Konrad Jachimstal}{211807} \and
  \studentinfo{ Patryk Janicki}{211951}
}

\title{Zadanie 1: Ekstrakcja cech, miary podobieństwa, klasyfikacja}

\begin{document}
\maketitle

\section{Cel}
{Celem zadania jest zbadanie wpływu ekstrakcji cech oraz wykorzystanych miar podobieństwa w procesie klasyfikacji tekstu. 
Klasyfikacja tekstów ma zostać zrealizowania z wykorzystaniem algorytmu najbliższych sąsiadów (KNN).}

\section{Wprowadzenie}
Do wykonania tego zadania niezbędne będzie skorzystanie z uczenia maszynowego.
Klasyfikacja tekstów, która odbywa się za pomocą algorytmu KNN. Polega ona na przypisaniu tekstu do odpowiedniej
kategorii. Odbywa się to na podstawie watrości poszczególnych wyekstrachowanych cech, które posiada każdy z tekstów.\\
Do ekstrakcji wykorzystywany jest znormalizowany zbiór słów badanego tekstu, oznaczony przez T. Normalizacja ma na
celu wyeliminowanie niepożądanych słów oraz sprowadzenie odmian słów o tym samym znaczeniu do jednego (określenie rdzenia słowa).

\subsection{Wykorzystane ekstraktory cech}

\subsubsection{Występowanie określonych słów kluczowych}
Występowanie określonych słów kluczowych w tekście,
nie uwzględnia liczności słów. Cecha to liczba słów kluczowych,
określająca ile słów kluczowych z całej puli wystąpiło w tekście.
Cecha przyjmuje wartości z zakresu [$0, n$] przy czym $n$ to liczba słów w tekście, zakładając, że
tekst składa się z tych samych słów kluczowych.

\subsubsection{Suma wystąpień słów kluczowych}
Suma wystąpień każdego słowa kluczowego w tekście. Cecha to suma wystąpień wszystkich słów kluczowych w tekście.
Cecha przyjmuje wartości od [$0, n$] gdzie $n$ liczba słów w tekście.

\begin{equation}
    F=S_{1} + ... + S_{n}
\end{equation}
gdzie:\\
\begin{description}
    \item $S$ - liczba wystąpień słowa kluczowego;
\end{description}


\subsubsection{Gęstość występowania słów kluczowych}
Gęstość wyrażona jako iloraz sumy wystąpień słów kluczowych i liczby wszystkich słów w tekście.
Cecha przyjmuje wartości liczbowe z zakresu [$0, 1$].
    \begin{equation}
      F=\frac{S}{L}
    \end{equation}
gdzie:\\
\begin{description}
    \item $S$ - suma wystąpień słów kluczowych;
    \item $L$ - liczba wszystkich słów w tekście;
\end{description}


\subsubsection{Średnia odległość słów kluczowych od początku tekstu}
Średnia odległość słów kluczowych wyrażona jako iloraz sumy odległości słów kluczowych od początku tekstu i liczby
słów kluczowych w tekście. Cecha to suma wartości liczbowych odzwierciedlających odległość danego słowa kluczowego
od początku tekstu, gdzie pojedyncze słowo odpowiada odległości równej jeden. Cecha przyjmuje wartości [$0, n$] gdzie n
jest maksymalną średnią z sumy odległości słów kluczowych od początku tekstu.
\begin{equation}
    F=\frac{S_{1} + ... + S_{n}}{L}
\end{equation}
gdzie:\\
\begin{description}
    \item $S$ - odległość słowa kluczowego od początku tekstu;
    \item $L$ -  liczba słów kluczowych w tekście;
\end{description}

\subsubsection{Liczba wszystkich słów}
Określenie liczby wszystkich słów występujących w tekście. 
Cecha przyjmuje wartości $n$ równą liczbie słów w tekście.
\begin{equation}
    F=|S|
\end{equation}
gdzie:\\
\begin{description}
    \item $S$ - to zbiór słów występujących w tekście;
\end{description}

\subsubsection{Rozproszenie słów kluczowych}
Rozproszenie słów kluczowych wyrażone jako iloraz sumy odległości pomiędzy słowami kluczowymi i iloczyn liczby
wszystkich słów w tekście oraz liczby słów kluczowych w tekście.

Możemy podjąć próbę i założyć, że większe nagromadzenie słów kluczowych na przestrzeni całego tekstu pozwala
stwierdzić, czy określone słowa kluczowe mają powiązanie z tematem tekstu, a nie stanowią wzmianki czy wstępu do
tekstu. Ekstrakcja ma pozwolić na określenie rozrzutu słów kluczowych w tekście. Cecha jest określona na podstawie
sumy odległości pomiędzy słowami kluczowymi z uwzględnieniem gęstości ich występowania. Przyjmuje wartości $[0, \infty]$.
\begin{equation}
    F=\frac{E_{d}}{N*N_{k}}
\end{equation}
gdzie:\\
\begin{description}
    \item $E_{d}$ - suma odległości pomiędzy słowami kluczowymi;
    \item $N$ - liczba wszystkich słów w tekście;
    \item $N_{k}$ - liczba słów kluczowych w tekście;
\end{description}

%\subsubsection{Występowanie określonych podciągów}
%\dots

\subsubsection{Częstość słów (ang. Term Frequency)}
Częstość słów wtyrażona jako iloraz liczby wytąpień słowa kluczowegoh przez ilość słów w dokumencie. Cecha to
częstotliwość wytępowania słowa kluczowego.
\begin{equation}
    F=\frac{K}{W}
\end{equation}
gdzie:\\
\begin{description}
    \item $K$ - liczba wystąpień danego słowa kluczowego w dokumencie;
    \item $W$ - liczba wszystkich słów w dokumencie;
\end{description}

\subsubsection{IDF (ang. inverse document frequency)}
IDF wyrażony jest jako iloraz liczby dokumentów poddawanych badaniu przez liczbę dokumentów w których dane słowo
kluczowe wystąpiło przynajmniej raz.

\begin{equation}
    F=log \frac{W}{D}
\end{equation}
gdzie:\\
\begin{description}
    \item $W$ - liczba wszystkich dokumentów;
    \item $D$ - liczba dokumentów w których wystąpiło słowo kluczowe;
\end{description}

\subsubsection{TF-IDF}
TF-IDF jest to stosunek częstości słów do częstości występowania danego słowa we wszystkich dokumentach.
Worażona jako iloczyn częstości słów (TF) oraz IDF.
\begin{equation}
    F={TF}\cdot{IDF}
\end{equation}
gdzie:\\
\begin{description}
    \item $TF$ - częstość słów;
    \item $IDF$ - częstość występowania na tle innych dokuemntów;
\end{description}

\newline \newline
{\color{blue}
We wprowadzeniu należy zaprezentować całą teorię potrzebną do realizacji
zadania (przy czym należy tu ograniczyć się wyłącznie do tego, co było
wykorzystane) tak aby osoba, która nigdy wcześniej nie zetknęła się z tą
tematyką, potrafiła zrozumieć dalszy opis. Część ta powinna wprowadzać
wszystkie wykorzystywane wzory, oznaczenia itp., do których należy się
odwoływać w dalszej części niniejszgo sprawozdania. Zamieszczony tu własny
opis teorii (a nie skopiowany!) należy poprzeć odwołaniami bibliograficznymi
do literatury zamieszczonej na końcu. }

\section{Opis implementacji}
{\color{blue}
Należy tu zamieścić krótki i zwięzły opis zaprojektowanych klas oraz powiązań
między nimi. Powinien się tu również znaleźć diagram UML  (diagram klas)
prezentujący najistotniejsze elementy stworzonej aplikacji. Należy także
podać, w jakim języku programowania została stworzona aplikacja. }

\section{Materiały i metody}

\subsection{Metryki - miara odległości}
Wykorzystane metryki służa do określenia odległości pomiędzy elementami tego samego zbioru w naszym przypadku tym
zbiorem będzie zbiór artykułów. Do obliczenia odległości pomiędzy dwoma artykułami na potrzeby algorytmu KNN zostały
wykorzystane metryki opisane poniżej.

\subsubsection{Metryka Euklidesowa}
Odległość Euklidesowa jest to odległość między dwoma wektorami określona jako pierwiastek kwadratowy sumy różnic między
wartościami podniesionymi do kwadratu, wyrażona wzorem:
\begin{equation}
    d(x,y)=\sqrt{\sum_{i=1}^{n}((x_{i}-y_{i})^{2})}
\end{equation}
gdzie:\\
\begin{description}
    \item $d$ - miara odległości;
    \item $x$, $y$ - wartości cech;
\end{description}

\subsubsection{Metryka Czebyszewa}
Odległość Czebyszewa jest to różnica pomiędzy znormalizowanymi cechami wartości obiektów, określona wzorem:
\begin{equation}
    d(x,y)=max_{i}|x_{i}-y_{i}|
\end{equation}
gdzie:\\
\begin{description}
    \item $d$ - miara odległości;
    \item $x$, $y$ - wartości cech obiektów;
\end{description}

\subsubsection{Metryka Manhattana}
Metryka Manhattana jest metryką podobną do metryki euklidesowej z tą różnicą, że odległość wyliczana jest
z bezwzględnych różnic pomiędzy wektorami. Odległość tę wyraża się wzorem:
\begin{equation}
    d(x,y)=\sum |x_{i}-y_{i}|
\end{equation}
gdzie:\\
\begin{description}
    \item $d$ - miara odległości;
    \item $x$, $y$ - wartości cech obiektów;
\end{description}

\subsection{Miary podobieństwa tekstów}
Miara podobieństwa tesktów to miara mówiąca o tym w jakim stopniu dany tekst $A$ jest podobny do tekstu $B$.

\subsubsection{Metoda n-gramów}
Metoda n-gramów określa w jakim stopniu łańcuch znaków $x$ jest podobny do łańcucha znaków $y$, na podstawie podciągów.
\begin{equation}
    sim_{n}(x,y)=\frac{1}{N-n+1}\sum_{i=1}^{N-n+1}h(i)
\end{equation}
gdzie:\\
\begin{description}
    \item $h(i)$ - przyjmuje 1 jeżeli dany podciąg występuje w łancuchu znaków $y$, w przeciwnym wypadku przyjmuje wartość 0;
    \item $N$ - liczba liter w słowie;
    \item $n$ - długość n-grama;
    \item $N-n+1$ - ilość n-elementowych podciągów w łańcuchu znaków;
\end{description}

\subsubsection{Uogólniona miara n-gramów}
Uogólniona miara n-gramów sprawdza podobieństwo słów w oparciu o podciągi o określonej długości. Wyrażona jest wzorem:

\begin{equation}
    u_{N}(x, y)=\frac{2}{N^{2}+N}\sum_{i=1}^{N(x)}\sum_{j=1}^{N(x-i+1)}h(i,j)
\end{equation}
gdzie:\\
\begin{description}
    \item $h(i,j)$ przyjmuje wartość 1 jeżeli dany podciąg ze słowa $x$ znajduje się w słowie $y$;
    \item $N(x),N(y)$ - ilość liter w słowach $x, y$, $N=max{N(x),N(y)}$;
    \item $\frac{N^{2}+N}{2}$ - ilość możliwych podciągów 1-elementowych do N-elementowych w słowie o długości $N$;
\end{description}

\subsubsection{Algorytm KMP (Knutha-Morrisa-Pratta)}
Algorytm KMP wyszukuje podany wzorzec $x$ w tekście $y$, jeżeli podany wzorzec zostaje znaleziony zwracana jest jego
pozycja w tekście. Wyrażony jest wzorem:
\begin{equation}

\end{equation}
gdzie:\\
\begin{description}
    \item
    \item
    \item
\end{description}


{\color{blue}
W tym miejscu należy opisać, jak przeprowadzone zostały wszystkie badania,
których wyniki i dyskusja zamieszczane są w dalszych sekcjach. Opis ten
powinien być na tyle dokładny, aby osoba czytająca go potrafiła wszystkie
przeprowadzone badania samodzielnie powtórzyć w celu zweryfikowania ich
poprawności (a zatem m.in. należy zamieścić tu opis architektury sieci,
wartości współczynników użytych w kolejnych eksperymentach, sposób
inicjalizacji wag, metodę uczenia itp. oraz informacje o danych, na których
prowadzone były badania). Przy opisie należy odwoływać się i stosować do
opisanych w sekcji drugiej wzorów i oznaczeń, a także w jasny sposób opisać
cel konkretnego testu. Najlepiej byłoby wyraźnie wyszczególnić (ponumerować)
poszczególne eksperymenty tak, aby łatwo było się do nich odwoływać dalej.}

\section{Wyniki}
{\color{blue}
W tej sekcji należy zaprezentować, dla każdego przeprowadzonego eksperymentu,
kompletny zestaw wyników w postaci tabel, wykresów itp. Powinny być one tak
ponazywane, aby było wiadomo, do czego się odnoszą. Wszystkie tabele i wykresy
należy oczywiście opisać (opisać co jest na osiach, w kolumnach itd.) stosując
się do przyjętych wcześniej oznaczeń. Nie należy tu komentować i interpretować
wyników, gdyż miejsce na to jest w kolejnej sekcji. Tu również dobrze jest
wprowadzić oznaczenia (tabel, wykresów) aby móc się do nich odwoływać
poniżej.}

\section{Dyskusja}
{\color{blue}
Sekcja ta powinna zawierać dokładną interpretację uzyskanych wyników
eksperymentów wraz ze szczegółowymi wnioskami z nich płynącymi. Najcenniejsze
są, rzecz jasna, wnioski o charakterze uniwersalnym, które mogą być istotne
przy innych, podobnych zadaniach. Należy również omówić i wyjaśnić wszystkie
napotakane problemy (jeśli takie były). Każdy wniosek powinien mieć poparcie
we wcześniej przeprowadzonych eksperymentach (odwołania do konkretnych
wyników). Jest to jedna z najważniejszych sekcji tego sprawozdania, gdyż
prezentuje poziom zrozumienia badanego problemu.}
\section{Wnioski}
{\color{blue}W tej, przedostatniej, sekcji należy zamieścić podsumowanie
najważniejszych wniosków z sekcji poprzedniej. Najlepiej jest je po prostu
wypunktować. Znów, tak jak poprzednio, najistotniejsze są wnioski o
charakterze uniwersalnym.}


\begin{thebibliography}{0}

\bibitem{art1} Adam Niewiadomski;
\textsl{Materiały, przykłady i ćwiczenia do przedmiotu
Komputerowe Systemy Rozpoznawania}; 21 września 2009;

\bibitem{art2} Isabelle Guyon, Steve Gunn, Masoud Nikravesh, Lofti A. Zadeh;
\textsl{Feature Extraction: Foundations and Applications}; Springer; 16 listopada 2008;

\bibitem{art3} David D. Lewis;
\textsl{Feature Selection and Feature Extract ion for Text Categorization}; University of Chicago; 26 września 1992;

\bibitem{art4} Knuth–Morris–Pratt algorithm;
\textsl{\url{https://en.wikipedia.org/wiki/Knuth-Morris-Pratt_algorithm}}

\bibitem{art5} B.S.Charulatha, Paul Rodrigues, T.Chitralekha, Arun Rajaraman;
\textsl{A Comparative study of different distance metrics that can be used in Fuzzy Clustering Algorithms}; 2013

\end{thebibliography}
\end{document}
